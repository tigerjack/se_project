\subsection{Revision history}
\begin{longtable}{| c | c | c | c |}
	% Some table settings
	\caption{\textbf{Revision History}}
	\label{tab:rev_history}
	\\ \hline
	
	% The table itself
	\textbf{Version} & \textbf{Date} & \textbf{Authors} & \textbf{Summary}\\ \hline
	1.0 & 22/01/2017& E. Migliorini, S. Perriello, A.Paglialonga & First release\\ \hline
\end{longtable}

\subsection{Purpose and scope}
This document represents the Project Plan Document for PowerEnjoy.
The main purpose of the document is to provide a methodological estimation of the project complexity, in order to provide a guidance for the definition of the required budget, the resources allocation and the schedule of the activities.

In section \ref{sec:project-size,-cost-and-effort-estimation}, we are going to use the Function Points (\ref{sec:function-points}) and COCOMO (\ref{sec:cocomo}) approaches together to provide an estimate of the expected size of the System in terms of lines of code and of the cost/effort required to actually develop it.

In section \ref{sec:task-scheduling}, we will reuse these figures to propose a possible schedule for the project that covers all activities from the requirements identification to the implementation and testing activities.

Finally, in section \ref{sec:risk-management} we will discuss the possible risks that the System could face during the various phase of the project and provide some general conclusions.

\subsection{Definitions, Acronyms, Abbreviations}
IFPUG: International Function Point User Group

\subsection{References}
\begin{itemize}
\item RASD, DD and IT Documents for the \textit{PowerEnJoy} application
\item 
Progressive Function Point Analysis Workbook \\
available at \texttt{https://sourceforge.net/projects/functionpoints/}
\item COCOMO II Model Manual\\
available at \texttt{http://sunset.usc.edu/research/cocomoii/Docs/modelman.pdf}
\end{itemize}