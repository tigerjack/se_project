A function point is a conceptual measure that express the amount of business functionality a software provides, based on what the end user request and receives.

The Function Points approach, originally defined in 1979 by Allan Albrecht, provides an estimation of the size of a project based.  The approach takes as inputs the functional user requirements of the software and each one is categorized into one of five types:
\begin{itemize}
	\item internal logic files (ILF) 
	\item external logic files (ELF)
	\item external inputs (EI)
	\item external outputs (EO)
	\item external inquiries (EQ)
\end{itemize} 

The 5 types of function points above, also known as Elementary Processes (EPs) can be grouped into 2 types of functions:

\begin{itemize}
	\item Inputs, Outputs and Queries all qualify as Transactional Functions and,
	\item Internal Files and External Files are distinguished as Data Functions
\end{itemize}

These groupings are helpful in determining the types of elements that each function is broken down into, to determine the complexity of the EP and ultimately the number of function points that should be awarded for a given EP.

A Transactional Function is broken down into DETs and FTRs, while a Data Function is broken down into DETs and RETs.

\begin{itemize}
	\item DET, Data Element Type is a unique user recognizable, non-repetitive field.
	\item FTR, File Type Referenced is a file type referenced by a transaction. An FTR must also be either an Internal or External file.
	\item RET Record Element Type is a user recognizable sub group of data elements within an Internal or External File.
\end{itemize}
   
Once the function is identified and categorized into a type, it is then assessed for complexity and assigned a number of function points.

\begin{longtable}{| c | c | c | c |}
	% Some table settings
	\caption{\textbf{Function Point Weights}} % Table caption
	\label{tab:fp_weights}%If later on you want to refer to this label, you can this label. 
	\\ \hline % end of row + new horizontal line
	
	% The table itself
	\textbf{Function Type} & \textbf{Low} & \textbf{Average} & \textbf{High}\\ \hline
	ILF & 7 & 10 & 15\\ \hline
	ELF & 5 & 7 & 10\\ \hline
	EI & 3 & 4 & 6\\ \hline
	EO & 4 & 5 & 7\\ \hline
	EI & 3 & 4 & 6\\ \hline	
\end{longtable}

\subsubsection{Internal Logic Files (ILF)}
PowerEnjoy has to store information related to various kind of entities in order to provide the required functionalities. All the homogeneous information are stored into different files or tables in a database. To ensure that the entities are uniquely identified inside a specific file or table we assign to each one an ID, which is unique inside the file or the table.
\bigskip

The first entities PowerEnjoy should track are surely the users; for this reason, we have a table for them storing for each one an email, a password, a social security number, a driving license number, a credit card number and a status (active/inactive). 
So, we can count 6 DETs, one for each of the data element identified above. Of course, we have to add the ID DET common to all the entities. Based on the information provided, we can thus judge that there will be only 1 RET ? meaning that all 7 DETs will be stored in a single file or Table in the database.
\bigskip

Secondly, we have to store information about areas: GPS latitude, GPS longitude, city and parking slots. The System has to manage two different kind of areas: a parking area and a charging area, the latter being an extension of the former. There are various kind of strategies to manage this kind of hierarchy at the data level; we choose to use the "single table strategy", in which the two classes of the hierarchy are mapped to a single table or file which has a discriminator column containing a value that identifies the subclass to which the instance represented by the record belongs. In our case, this discriminator is a boolean condition that is set to true if the specified record is a charging area. In addition to this field, for a charging area, we also have to add another field to store the number of charging slots associated to the charging area.\\
So, in the end we can come up with 1 RET and 7 DETs.
\bigskip

Another important piece of information is associated to the cars managed by the System. For each car, we have to store plate number, GPS latitude, GPS longitude, battery level and the status (available, unavailable, reserved or in use). We must also have a field to check if the car is plugged to a socket of a charging area. Finally, we have to know in which area the car is; for this reason, we have a field that has the identifier of an area. Thus, for this kind of data, we came up with 1 RET and 8 DETs.
\bigskip

Closely related to a car we also need a table containing all the damages. A damage, for this reason, has a mandatory identifier of the car on which the damage is. Apart from this, the table has other fields to help the identification of the damage: a text containing the description and if the specific damage is a major damage or not. It also stores two different timestamps related to when the damage was detected and when (optionally) it was solved. Of course, we have also a boolean flag that indicates if a damage has been solved or not. For this reason, we count 1 RET and 8 DETs.
\bigskip

For the main functionalities provided by our System we have to store other two different kind of homogeneous data: the first one for the reservations and the second one for the drivings.
First of all, each reservation has a reference to both the ID of the user which has made the reservation and the ID of the car being reserved. Also, we store the time on which the reservation was made and the time on which it was concluded. Finally, we have a boolean flag to know if the reservation is currently active.
We came up with 1 RET and 5 DETs.
\bigskip

The driving table is very similar to the previous one, storing a reference to both the ID of the user which driving the car, the ID of the car being driven, the time on which the drive started, the time on which it was concluded and the active flag. Apart from the previous fields, we have to store other information relevant for the evaluation of the fee applied to the user: three flag to know if the drive has to be applied a discount (the user has taken other passengers, the user left the auto with an high battery, the user plugged the car into a socket at the end of the ride) and other two for a surcharge (if the user left the auto with a low battery or if the car was left far from a charging area). In the end, we have 1 RET and 11 DETs.
\bigskip

Another homogeneous kind of data stored is related to the banking operations managed by our System. A banking operation can be related to an expired reservation or to a driving. For this reason, we have two optional data elements: the first one refers to the ID of a reservation, the last one to the ID of a driving. Of course we must also store the final fee of the specific banking operation, if it has been paid and if it has been processed. Thus, we have 1 RET and 6 DETs.
\bigskip

In the ILF we must surely count the various configuration files used by the System to define the amount of each banking operation. This kind of data relies on many different variables:
\begin{enumerate}
	\item the fee per driving minutes contains how much a user should pay for each minute of drive
	\item fee per expired reservation represents how much a user should pay for a reservation which is expired
	\item passengers discount percentage represents the percentage of the discount to be applied in the case in which the user picked up other passengers during the drive
	\item passengers number for discount is the data elements saying how many passengers a user should have picked up in order to qualify for the passengers discount
	\item passengers time for discount is the data elements saying the minimum amount of time the passengers should be in the car in order to qualify for the passengers discount
	\item high battery discount percentage represents the percentage of the discount to be applied in the case in which the user left the car with an high battery percentage at the end of the ride
	\item high battery percentage for discount  is the data elements saying the minimum battery percentage level requested to apply an high battery discount
	\item plugged car discount percentage represents the percentage of the discount to be applied in the case in which the user connected the car plug to a socket of a charging area at the end of the ride
	\item plugging car time indicates the maximum time the System waits until the user connects the plug, after which the discount is not applied anymore
	\item away from charging area surcharge percentage represents the percentage of the surcharge to be applied in the case in which the user, at the end of the ride, left the Car away from a charging area
	\item away from charging area meters represent how much is far away, id est the minimum distance between the car and the nearest charging area used to apply the surcharge at the previous point
	\item low battery surcharge percentage represents the percentage of the surcharge to be applied in the case in which the user, at the end of the ride, left the Car with a low battery percentage
	\item low battery percentage for surcharge represents the maximum percentage of the battery that has to be considered as low battery for the surcharge at the previous point.
\end{enumerate}
We don't need an ID for this kind of data, so we can easily conclude that we have 1 RET and 13 DETs.
\bigskip

The last kind of data we should manage is related to the configuration of other parameters related to the cars, namely:
\begin{enumerate}
	\item locate car nearby range, id est the maximum range between an user and the available cars that the System should use;
	\item unlock car nearby range, id est the maximum range between an user and the car he/she has reserved inside which he/she can unlock the car;
	\item battery percentage available, id est the minimum percentage level that the System should use to classify a car as available.
\end{enumerate}
We can easily came up with 1RET and 3 DETs.
\bigskip

At the end of the ILF analysis, basing on the value of the tables \ref{tab:fp_weights} and \ref{tab:ilf_complexity_matrix}, we came up with the value indicated in the table \ref{tab:ilf_weights}.

\begin{longtable}{| c | c | c | c |}
	% Some table settings
	\caption{\textbf{Complexity matrix}} % Table caption
	\label{tab:ilf_complexity_matrix}%If later on you want to refer to this label, you can this label. 
	\\ \hline % end of row + new horizontal line
	
	% The table itself
	\textbf{RETs} &	\multicolumn{3}{c|}{\textbf{DETs}} \\ \hline
	  & 1-19 & 20-50 & 51+\\ \hline 
	1 & L & L & A\\ \hline 
	2 to 5 & L & A & H\\ \hline 
	6 or more & A & H & H \\ \hline 
	
\end{longtable}

\begin{longtable}{| c | c | c |}
	% Some table settings
	\caption{\textbf{Weights}} % Table caption
	\label{tab:ilf_weights}%If later on you want to refer to this label, you can this label. 
	\\ \hline % end of row + new horizontal line
	
	% The table itself
	\textbf{Function Type} & \textbf{Complexity} & \textbf{FPs}\\ \hline
	User & Low & 7\\ \hline
	Car & Low & 7\\ \hline
	Area & Low & 7\\ \hline
	Damage & Low & 7\\ \hline
	Reservation & Low & 7\\ \hline
	Driving & Low & 7\\ \hline
	Banking & Low & 7\\ \hline
	Fees & Low & 7\\ \hline
	Cars & Low & 7\\ \hline
\end{longtable}

\subsubsection{External Logic Files (ELF)}	
Those are data used and referenced by our System but not generated and maintained by it.
	e.g. External Systems 

\subsubsection{External Input (EI)}
	elementary operations to elaborate data coming from extern environment
	e.g. calls to our app by clients: register, confirm, login/out, reserve cars, unlock cars, drive car, end ride, enable money saving

\subsubsection{External Output (EO)}
	elem ops that generates data for the ext environm. includes elaboration of data from logic files
	e.g. notify charge to user, notify company of damage, notify 

\subsubsection{External Inquiry (EQ)}
	elem ops that involves input and output. w/out significant elaboration of data from logic files
	e.g. show areas, show available cars, 