% Counter to use for UseCaseId
\newcounter{UseCaseIdCounter}

% It starts from 0 by default, so I've incremented it.
\stepcounter{UseCaseIdCounter}

\begin{longtable}{|p{0.2\linewidth} p{0.8\linewidth}|}
	% Some table settings
	\captionsetup{labelformat=empty} % To not show Table 1, Table 2 under the table
	\caption{\textbf{Register Account}} % Table caption
	\label{UC_Register}%If later on you want to refer to this label, you can this label. 
	\\ \hline % end of row + new horizontal line
	
	% The table itself
	\textbf{ID} & UC\theUseCaseIdCounter \\ \hline
	\textbf{Description} & The Visitor wants to create an account for the Car-Sharing Service. \\ \hline
	\textbf{Actors} & \emph{Visitor}.\\ \hline
	\textbf{Pre-Conditions} & The \emph{Visitor} connects to the \emph{Company's Car-Sharing}} WebSite/Application. \\ \hline
	\textbf{Flow of events} & 
		\begin{enumerate}
			\item The \emph{Visitor} selects the function \textit{\textquotedblleft{Sign Up}\textquotedblright}.
			\item The \emph{System} returns a form to enter all the required data: Name, Surname, Birth date, ID Card Number, Driving License number and Credit Card number. It also asks for an email address and a password which will be used for the future logins.
			\item The \emph{Visitor} fills the form with all the required information.
			\item The \emph{System} stores the request together with all the data provided with it, generates a random activation URL and asks the \emph{Mail System} to forward his/her URL to the email address of the \emph{Visitor}.
		\end{enumerate}	 \\ \hline
	\textbf{Post Conditions} & The \emph{Mail System} sends the activation URL to the \emph{Visitor}'s email provided in the registration form. \\ \hline
	\textbf{Exceptions} & 
		\begin{itemize}
		\item The \emph{System} recognizes invalid or missing data in the form compiled by the emph{Visitor}and informs him/her of the error. The flow of events restarts from point 1.
		\item The Visitor inserts in the form a Social Security Number, or ID Card Number, or Driving License number, or Email Address, which is already present in the System. The System shows an error message saying that some of those credentials were already been inserted into the System for another account. The flow of events restarts from point 1.
		\end{itemize} \\ \hline
\end{longtable}

\stepcounter{UseCaseIdCounter}

\begin{longtable}{|p{0.2\linewidth} p{0.8\linewidth}|}
	% Some table settings
	\captionsetup{labelformat=empty} % To not show Table 1, Table 2 under the table
	\caption{\textbf{Activate Account}} % Table caption
	\label{UC_Activate}%If later on you want to refer to this label, you can use this label. 
	\\ \hline % end of row + new horizontal line
	
	% The table itself
	\textbf{ID} & UC\theUseCaseIdCounter \\ \hline
	\textbf{Description} & The Visitor wants to activate his/her account. \\ \hline
	\textbf{Actors} & \emph{Visitor}.\\ \hline
	\textbf{Pre-Conditions} & The \emph{Visitor} has received the activation URL on his/her mail box. \\ \hline
	\textbf{Flow of ev-ents} & 
	\begin{enumerate}
		\item The \emph{Visitor} opens the received activation URL.
		\item The \emph{System} acknowledges that the Visitor has arrived in his/her activation Web Page and activates his/her account.
	\end{enumerate}	 \\ \hline
	\textbf{Post Conditions} & The \emph{Visitor} is now become an \emph{User} which can access the \emph{System} using the credentials (Email, password) he provided during the registration phase. \\ \hline
	\textbf{Exceptions} & 
	\begin{itemize}
		\item The Activation URL expires after 10 days it has been generated. The Visitor?s data are cancelled from the System and the Visitor will have to perform the Registration (UC1) again.
	\end{itemize} \\ \hline
\end{longtable}

\stepcounter{UseCaseIdCounter}

\begin{longtable}{|p{0.2\linewidth} p{0.8\linewidth}|}
	% Some table settings
	\captionsetup{labelformat=empty} % To not show Table 1, Table 2 under the table
	\caption{\textbf{Log In}} % Table caption
	\label{UC_Login}%If later on you want to refer to this label, you can use this label. 
	\\ \hline % end of row + new horizontal line
	
	% The table itself
	\textbf{ID} & UC\theUseCaseIdCounter \\ \hline
	\textbf{Description} & The Visitor wants to log in the System. \\ \hline
	\textbf{Actors} & \emph{Visitor}.\\ \hline
	\textbf{Pre-Conditions} & The \emph{Visitor} connects to the Company's \emph{Car-Sharing WebSite/Application} \\ \hline
	\textbf{Flow of events} & 
	\begin{enumerate}
		\item The \emph{Visitor} selects the function \textit{\textquotedblleft{Login}\textquotedblright} .
		\item The \emph{System} shows the \emph{Visitor} a login form, asking him to insert the email and password provided in the registration form.
		\item The \emph{Visitor} inserts the pair (Email,Password) used during the registration phase and selects the function \textit{\textquotedblleft{Log me in}\textquotedblright}
	\end{enumerate}	 \\ \hline
	\textbf{Post Conditions} & The \emph{System} verifies the existence of an account associated with that pair (Email,password) and logs the \emph{Visitor} in. The \emph{Visitor} has now become \emph{User}  \\ \hline
	\textbf{Exceptions} & 
	\begin{itemize}
		\item The Activation URL expires after 10 days it has been generated. The Visitor?s data are cancelled from the System and the Visitor will have to perform the Registration (UC1) again.
	\end{itemize} \\ \hline
\end{longtable}
\stepcounter{UseCaseIdCounter}

\begin{longtable}{|p{0.2\linewidth} p{0.8\linewidth}|}
	% Some table settings
	\captionsetup{labelformat=empty} % To not show Table 1, Table 2 under the table
	\caption{\textbf{Log Out}} % Table caption
	\label{UC_Logout}%If later on you want to refer to this label, you can use this label. 
	\\ \hline % end of row + new horizontal line
	
	% The table itself
	\textbf{ID} & UC\theUseCaseIdCounter \\ \hline
	\textbf{Description} & The User wants to log out from the System. \\ \hline
	\textbf{Actors} & \emph{User}.\\ \hline
	\textbf{Pre-Conditions} & The \emph{User} is logged in the \emph{System} \\ \hline
	\textbf{Flow of events} & 
	\begin{enumerate}
		\item The \emph{User} selects the function \textit{\textquotedblleft{Log out}\textquotedblright} .
		\item The \emph{System} performs the \emph{User}'s logout.
	\end{enumerate}	 \\ \hline
	\textbf{Post Conditions} & The \emph{System} shows the confirmation of the logout to the \emph{User}\\ \hline
	The \emph{User} is now not able to use the \emph{System} functionalities dedicated to Users anymore (until he logs in again). \\ \hline
	\textbf{Exceptions} & 
	\begin{itemize}
	\end{itemize} \\ \hline
\end{longtable}
\stepcounter{UseCaseIdCounter}

\begin{longtable}{|p{0.2\linewidth} p{0.8\linewidth}|}
	% Some table settings
	\captionsetup{labelformat=empty} % To not show Table 1, Table 2 under the table
	\caption{\textbf{Locate Available Cars}} % Table caption
	\label{UC_LocateCars}%If later on you want to refer to this label, you can use this label. 
	\\ \hline % end of row + new horizontal line
	
	% The table itself
	\textbf{ID} & UC\theUseCaseIdCounter \\ \hline
	\textbf{Description} & The User wants to locate the avialable Cars. \\ \hline
	\textbf{Actors} & \emph{User}.\\ \hline
	\textbf{Pre-Conditions} & The \emph{User} is logged in the \emph{System} \\ \hline
	\textbf{Flow of events} & 
	\begin{enumerate}
		\item The \emph{User} selects the function \textit{\textquotedblleft{Locate Cars}\textquotedblright} .
		\item The \emph{System} shows a text box asking the \emph{User} to provide an address near which he/she would like to see the \emph{Cars} whose state is \textit{Available}.
		\item The \emph{User} inserts the desired address and selects the \textit{\textquotedblleft{Locate}\textquotedblright} function.
	\end{enumerate}	 \\ \hline
	\textbf{Post Conditions} & The \emph{System} shows the confirmation of the logout to the \emph{User}\\ \hline
	The \emph{User} is now not able to use the \emph{System} functionalities dedicated to Users anymore (until he logs in again). \\ \hline
	\textbf{Exceptions} & 
	\begin{itemize}
	\end{itemize} \\ \hline
\end{longtable}
%%%%%OTHER USE CASES!!
USE CASE : Locate available cars.
Id:  UC5

Description
The User wants to locate a Car.

Actors
User.

Pre-Condition
The User must be logged into the System.

Flow of events

1 The User selects the function ?Locate Cars?.
2 The System shows a text box asking the User to provide an address near which he would like to see the Cars whose state is Available.
3 The User inserts the desired address and selects the ?Locate? function.

Post-Condition
The System shows the User a map containing all the Cars, whose state is Available, which are inside a 5KM distance range from the provided address or User?s GPS position.
Alternative flow of events
3a 	The User selects the function ?Near me? instead of inserting a specific address and sends his/her GPS Coordinates to the System.

Exceptions :

4b	The System does not find the inserted address and informs the User. The Flow of Events starts from point 1.
4c	There are no available Cars in the specified address/User?s Position. The System informs the User. The Flow of Events start from point 1.

		
USE CASE : Reserve available car.


Id:  UC6


Description
The User wants to reserve a Car.

Actors
User

Pre-Condition
The User must be logged into the System and the System must have found cars when the User activated the ?Locate available cars? function. 


Flow of events
1. The User chooses a specific Car between the showed ones in the map.
2. User selects the function ?Reserve this Car?.

Post-Condition
The System stores the Reservation of the Car, changing the Car status in Reserved.
The System activates a countdown of 1 hour during which the User will have the possibility to unlock the reserved Car.


USE CASE : Unlock Car.
ID : UC9

DESCRIPTION : The User wants to ask the System to open the doors of the Car in order  to enter it.
PARTECIPATING ACTORS : User  
PRE-CONDITION : The User must be logged in the System and must have reserved a car. 
FLOW OF EVENTS : 
1 The User activates the function ?Unlock Car?.
2 The User  sends his/her GPS coordinates to the System; 
3 The System checks that the GPS coordinates of the specific User ?s Reserved Car and the GPS coordinates of the User himself are in a 15 meters distance range. 
EXCEPTIONS :
If one hour has passed since the reservation has been done and the User didn?t unlock the Car because he wasn?t in the 15 meters distance range or didn?t activate this function :
1. the reservation expires and the User cannot unlock the car anymore (unless with another reservation). 
2. The System changes the Car status to Available.
3. The System communicates to the Banking System the amount of money (corresponding to the fee of 1 EUR) to charge to the User.
4. The System now allows the User to perform another reservation. 
POST-CONDITION :  
The System has verified that User is nearby the car (inside the specified distance range) and unlocks the Car doors. 
The System now changes the Car status to In Use and sets the Plugged Field False. 
The User enters the Car.



USE CASE : Drive Car. 
ID : UC10				
DESCRIPTION  : The User starts to drive the reserved Car.
PARTECIPATING ACTORS : User 
PRE-CONDITION : The User has unlocked the doors of the car and entered the Car.
FLOW OF EVENTS : 
1 The User starts the engine of the Car.
2 The System starts the Ride Timer which indicates the time usage of the Car.
3 [Extension Point UC11].
4 [Extension Point UC14].
5 The System calculates the current fee charged to the User (calculated as a given amount of money per minute multiplied by the minutes of the Ride Timer) while showing it on the on-board screen.
POST-CONDITION :  
The User drives the Car. 




USE CASE : Drive with Passengers.  <<extends UC10>>
ID : UC 11
DESCRIPTION : The User picks up passengers to share the ride with.
PARTECIPATING ACTORS : User
PRE-CONDITION : The User is driving his/her reserved Car.

FLOW OF EVENTS : 
1. The User picks up the desired passengers.
2. The Car detects the number of passengers.

POST-CONDITION :  
The User is sharing the ride with his/her passengers.
The System stores the number of passengers in the ride and if they stayed in the Car for at least 3 minutes.

USE CASE : End ride.
Id : UC12
DESCRIPTION : The User ends the ride and the System processes the fee.
PARTECIPATING ACTORS : User 
PRE-CONDITION : The User parks the Car in one of the Parking Areas.
FLOW OF EVENTS : 
1. The User exits the Car.
2. The System verifies that no one is in the car.
3. The System checks the battery status. 
4. The System checks, using the GPS coordinates, if the User has left the Car within a 3KM distance range from the nearest Charging Area.
5. The System checks if the User drove with passengers (UC11).
6. [Extension Point UC13].
POST-CONDITION :  
The System locks the doors of the Car and sets it as Available.
The System communicates to the Banking System the final fee to charge to the User.
ALTERNATIVE FLOW OF EVENTS:
7a 	The battery status is higher than 50%, the User didn?t or did take at least 2 passengers with him for at least 3 minutes (UC11) , didn?t leave the Car at more than 3KM from the nearest Charging Area, didn?t plug the Car (UC13), hence the System applies a 20% discount on the fee of the last ride and communicates to the Banking System the fee which will be charged to the User.
7b The User did plug the Car (UC13), the battery status is higher than or equal to 20%, he/she didn?t or did take at least 2 passengers with him for at least 3 minutes (UC11), hence the System applies a 30% discount on the fee of the last ride and communicates to the Banking System the fee which will be charged to the User.
7c The User did plug the Car (UC13), the battery status is lower than 20%, he/she didn?t or did take at least 2 passengers with him for at least 3 minutes (UC11), didn?t plug the Car (UC13), hence the System doesn?t apply any discount or surcharge on the fee of the last ride and communicates to the Banking System the fee which will be charged to the User.
7d The User didn?t plug the Car (UC13), the battery status is higher than 50%, he/she did or didn?t take at least 2 passengers with him for at least 3 minutes (UC11), did leave the Car at more than 3KM from the nearest Charging Area, hence the System applies a 10% surcharge on the fee of the last ride and communicates to the Banking System the fee which will be charged to the User.
7e 	The battery status is between 20% and 50% (included), the User did take at least 2 passengers with him for at least 3 minutes (UC11) , didn?t leave the Car at more than 3KM from the nearest Charging Area, didn?t plug the Car (UC13), hence the System applies a 10% discount on the fee of the last ride and communicates to the Banking System the fee which will be charged to the User.
7f The battery status is lower than 20%, the User did take at least 2 passengers with him for at least 3 minutes (UC11) , did or didn?t leave the Car at more than 3KM from the nearest Charging Area, didn?t plug the Car (UC13), hence the System applies a 20% surcharge on the fee of the last ride and communicates to the Banking System the fee which will be charged to the User.
7g The battery status is between 20% and 50% (included), the User did take at least 2 passengers with him for at least 3 minutes (UC11) , did leave the Car at more than 3KM from the nearest Charging Area, hence the System applies a 20% surcharge on the fee of the last ride and communicates to the Banking System the fee which will be charged to the User.
7h The battery status is lower than 20%, the User didn?t take at least 2 passengers with him for at least 3 minutes (UC11) , did or didn?t leave the Car at more than 3KM from the nearest Charging Area, didn?t plug the Car (UC13), hence the System applies a 30 % surcharge on the fee of the last ride and communicates to the Banking System the fee which will be charged to the User.
7j The battery status is higher than 50%, the User did or didn?t take at least 2 passengers with him for at least 3 minutes (UC11) , did leave the Car at more than 3KM from the nearest Charging Area, didn?t plug the Car (UC13), hence the System applies a 10 % surcharge on the fee of the last ride and communicates to the Banking System the fee which will be charged to the User.
7k The battery status is between 20% and 50% (included), the User didn?t take at least 2 passengers with him for at least 3 minutes (UC11) , did leave the Car at more than 3KM from the nearest Charging Area, hence the System applies a 30% surcharge on the fee of the last ride and communicates to the Banking System the fee which will be charged to the User.

EXCEPTIONS : 
The Ride ends and the Car stops moving when the battery status reaches 0% of the full capacity or when the Car detects a major damage. 

USE CASE : Plug the Car.  ( <<extends UC12>> )
ID : UC13
PARTECIPATING ACTORS : User 
PRE-CONDITION : The User has parked the Car in one of the Charging Areas designated by the System. 
FLOW OF EVENTS :
1. The User plugs the Car into the socket of the Charging Area. 
2. The System detects that the Car is plugged within 2 minutes since the User got off the Car.
POST-CONDITION :  
The battery of the Car is charging and the System stores the User?s action for possible discounts.
The System sets the Car Plugged Field True.
USE CASE : Enable Money Saving Option. ( <<Extends UC10>> )
ID : UC14
PARTECIPATING ACTORS : User 
PRE-CONDITION : The User enables the Money Saving Option.
FLOW OF EVENTS : 
1. The System asks the User the address of his final destination showing a text box where to insert the address.
2. The User provides the address to the System.

POST-CONDITION :  
The System computes an algorithm which takes in consideration the distribution of the cars in the city, the final destination of the User and the availability of power plugs in the Charging Area. The result of this algorithm will be sent to the User providing him the address of the Charging Area where to leave the Car. 
(The User will still have to plug the Car in order to have a discount!)

EXCEPTIONS : 
If the Socket of the Charging Area has no more available plugs , the System informs the User and the Flow of Events starts from point 1.