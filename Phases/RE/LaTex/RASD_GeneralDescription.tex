This section will give a broad overview of the whole system under development. It will explain how the system interacts with other systems and introduce the main functionality of it.

It will also describe the main stakeholders of the system and the functionalities available for the users of the system, detailing all the informations relevant to clarify their needs.

At last it will present the constraints and assumptions made for the system under development.

\subsection{Product Perspective}

PowerEnjoy is an ex-novo system that will be used to support all the functionalities required by the Company.

The structure of the main system is composed by a central elaborator (server), that will be responsible for all necessary operations, including but not limited to the communication with the Database and all the external interfaces. For this reason, PowenEnjoy system has to manage the interaction with all these kind of interfaces, through a set of shared protocols and APIs.

End users will access the server through an Internet connection, using the Application, that will be part of the PowerEnjoy system. For this reason, the system should also be able to implement the Application Layer of the Internet Protocol Suite.

Furthermore, Users should be able to communicate to the system their GPS position through the Application in order to unlock the previously reserved Car.

At last, the system should be able to communicate with the wide variety of sensors and devices placed inside the Cars, in order to know, in every moment, their position, the status of their battery, their damages, the connection to an electrical socket and the number of seats occupied. 



\subsection{Product Functions}
With the Application, Users will be able to register to the system, locate all the Car available in a certain range, reserve them and unlock the reserved car when they are nearby it. At the end of the drive period, the system, basing on the time usage period of the Car, will evaluate the Fee and notifies both the User and the Banking System of the total amount.

The system is also aimed to promote good behaviour of the Users through various kind of discounts. For this reason, the final Fee is not fixed, but it will vary depending on various User actions, specified in further sections.

%TODO format requirements
\newcounter{RequirementsCounter}
\stepcounter{RequirementsCounter}

\begin{center}
  \begin{longtable}{|p{.2\textwidth}|p{.12\textwidth}|p{.58\textwidth}|}
    \hline
    \textbf{Need} & \textbf{Priority} & \textbf{Function} \\ \hline
    Registration & ? & Allows a Visitor to register their credentials, becoming a User \\ \hline
    Login & ? & Lets a User access the system \\ \hline
    Find Nearby Cars & ? & Shows Available Cars in a location \\ \hline
    Reservation & ? & Allows a user to set a nearby Car as Reserved \\ \hline
    Proximity Unlock & ? & Unlocks the car when the User who Reserved it is close \\ \hline
  \end{longtable}
\end{center}

\begin{center}
  \begin{longtable}{|p{.2\textwidth}|p{.12\textwidth}|p{.58\textwidth}|}
    \hline
    \textbf{Need} & \textbf{Priority} & \textbf{Function} \\ \hline
    Computation & ? & Computes the Fee to be paid, including Discounts and Increases \\ \hline
    Payment & ? & Charges the user for the computed Fee \\ \hline
    Detect Parking & ? & Detects whether a Car has been left in a Parking Area \\ \hline
    Detect Charging & ? & Detects whether a Car is Plugged to a Charging Area \\ \hline
    Administrative Access & ? & Allows an administrator to modify the information in the Database or parametres of the System \\ \hline
  \end{longtable}
\end{center}

\subsection{User Classes and Characteristics}
\begin{center}
  \begin{tabular}{|p{.2\textwidth}|p{.39\textwidth}|p{.39\textwidth}|}
    \hline
    \textbf{Name} & \textbf{Description} & \textbf{Responsibility} \\ \hline
    Visitor & A person who needs to undergo Registration to become a User & Performs a Registration \\ \hline
    User & Someone who is registered in the system and can access its functionalities & Can find, reserve and rent Cars \\ \hline
    Administrator & A specialized worker for the Company & Can modify the Database and the application's various parameters \\ \hline
  \end{tabular}
\end{center}

\subsection{Constraints}
\subsubsection{Operating Environment}\label{OE}
Users and Visitors will access the System through a smartphone application. The smartphones used to access the application will need to have a sufficiently advanced hardware in order to run it, and the application will require both an Internet connection and a GPS signal. The mobile application will, however, only be required in order to run the Find Nearby Cars, Reservation and Proximity Unlock functions.

The Cars will also need to be modified by adding a device capable of getting GPS signal and remotely communicate with the central System.

The central System will need a suitably powerful elaborator to run on.

\subsubsection{Design and Implementation Constraints}
The system will employ the HTTP protocol in order to have the client software, both the software application and the one installed on the Cars, communicate with the central System. As mentioned above, a network connection will be necessary. In order to maximize availability, it is advised to use the best available hardware.

User ID and password will be required in order to access the System. In order to register as a User, a Visitor needs to undergo Registration. 

\subsection{Assumptions and Dependencies}
\begin{enumerate}
	\item The User can only have one Account at time.
	\item The Company can decide at any time to block an User from the access to the system (f.e. for improper behavior, unpaid bill, ...). The system does not provide this functionality, but has to cope with it.
	\item In order to use the system functionalities, the User should provide valid Credit Card informations. The Credit Card should not necessarily belong to the same User registering.
	\item We trust the data provided by the User during his/her registration on the System.
	\item The Database in which the Cars, Parking Areas, Charging Areas, Users,etc, are stored is managed by an external Company, which is responsible for its security, reliability and availability.
	\item There's an external company which manages the employees.
	\item The Car has a set of sensors that can detect, in every moment, its position, the status of its battery, the status of the engine, its damages, the connection of its plug to an electrical socket and the number of seats occupied.
	\item The User always enters the Car when he/she unlocks its doors.	
	\item After a Car is Plugged, it will not be maliciously unplugged by the User himself/herself or by other people.
	\item After the doors of the cars are unlocked by the User, he/she always enters the Car, ignites the engine and leave the Parking/Charging Area.
	\item An User can park/stop the Car everywhere and leave the Car at anytime. However, the system will end the ride (i.e. stop charging the User) only if he/she turns the engine off inside a Parking Area.
	\item When the User gets Passengers, the corresponding discount on the User's fee will be applied only if the passengers stay in the Car for more than 3 minutes.	
	\item When a User will park the Car inside a Parking Area, it will always correctly use one of the free slots.
	\item As soon as the Car battery status gets below 20\% of the full capacity AND the Car isn't in a Charging Area AND the Car state isn't \textit{In Use} OR \textit{Plugged}, there's always an Employee that immediately reaches the Car and recharges it on site; in the meanwhile the Car status is \textit{Unavailable}.	
	\item When the Car battery status reaches the 0\% of the full capacity the Car stops working and is immediately set as \textit{Unavailable}.
	If the Car status is \textit{Unavailable}, the Car will be reached by an Employee to consider if the Car needs to be taken in the Company's workshop for repairs or just needs to be recharged.
	\item The Car has the ability to detect if it has been damaged. 
	\item If the Car status is \textit{In Use} when a \textit{Minor damage} is detected, the Car status will be set to \textit{Unavailable} at the end of the ride; if a \textit{Major damage} is detected the Car status is immediately set to \textit{Unavailable}. In both cases an employee will reach the car and cope with the damages, deciding if the Car can be immediately used again (sets it to \textit{Available}) and/or if the User should pay for the damages. 
	\item A car which is \textit{Available} or \textit{Plugged} can be set as \textit{Unavailable} in every moment by an Employee. Our system do not offer this functionality, but has to cope with it.
	\item A car which is \textit{Unavailable} can be set to \textit{Available} in every moment by an Employee. Our system do not offer this functionality, but has to cope with it.
	\item If the Car has been left out of a Parking Area there will always be an employee which immediately reaches it, recharges it and move it to a Charging Area. 
	\item Every fine received by the Company for improper use of the Car will be managed by the Company.
\end{enumerate}

\clearpage