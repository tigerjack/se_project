\section{Introduction}

\subsection{Purpose}
This paper represents the \textbf{R}equirement \textbf{A}nalysis and \textbf{S}pecification \textbf{D}ocument of the \textit{System To Be}, which will implement the \emph{PowerEnJoy Car-Sharing} Service. This document aims at explaining the functionalities of the System in terms of Functional Requirements, NonFunctional Requirements and Special Requirements, represented using both diagrams and natural language.
 
The below is a comprehensive list of functionalities provided by the system, that actually translates to a list of goals that the system should reach.

\begin{enumerate}
	\item The system should allow the registration of the Visitors with their credentials and payment informations.
	\item The system should allow all Users to use all the functionalities reserved to them.
	\item The system should be able to give each User the list of all the available cars in a range of 5KM from his/her GPS position.
	\item The system should allow each of its Users to reserve a Car whose state is Available.
	\item If an User has reserved a Car and he/she not unlock it within 1 hour from the reservation, the system set the car state as Available, the reservation expires and the user pays a fixed Fee of 1 EUR.  
	\item The system should allow each User to unlock a previously reserved Car when he/she is in a distance range of 15 meters from the same Car.
	\item The system should allow each User to drive a Car which he/she has previously unlocked, .
	\item The system should be able to know the time usage of the minutes.
	\item The system should allow each User to end the ride in a Parking Area.
	\item If the system detects the user took at least two other passengers onto the car, the system applies a discount of 10\% on the last ride. 
	\item If a car is left with no more than 50\% of the battery empty, the system applies a discount of 20\% on the last ride. 
	\item If a car is left at special parking areas where they can be recharged and the user takes care of plugging the car into the power grid, the system applies a discount of 30\% on the last ride. 
	\item If a Car is left at more than 3 KM from the nearest power grid station or with more than 80\% of the battery empty, the system charges 30\% more on the last ride to compensate for the cost required to recharge the car on-site.
	\item If the user enables the money saving option, he/she can input his/her final destination and the system provides information about the Charging Area where to leave the Car in order to get a Discount on the total Fee. The Charging Area is determined by the system to ensure a uniform distribution of Cars in the city and depends both on the destination of the User and on the availability of Sockets at the selected Charging Area. 
\end{enumerate}

\subsection{Intended Audience}
This document is addressed to all the stakeholders involved in the \emph{PowerEnJoy} project. This includes, but it is not limited to, the development committee, product designers and engineers, quality assurance, who will decide if the requirements described in this document have met the intended system requirements.

\subsection{Product Scope}
The aim of the \emph{PowerEnJoy} project is to provide a \textit{Car-Sharing} Service which implements electric-powered cars only.
This system will have to interface the Cars, Charging Areas, allowing Users to reserve, unlock, drive and park Cars, finally charging them the cost of the ride. 
The \textbf{System} will keep track of \textbf{Car}s' position, battery level, possible damages, plugging state.

\subsection{Definitions, Acronyms and Abbreviations}
\subsubsection{Business terms glossary}
\begin{itemize}
	\item \emph{Account}
	An account is a virtual representation of a User. The System can read and store information about a User by editing their Account.
	
	\item \emph{Battery}
	A Battery powers a Vehicle. The charge state of the Battery can be anywhere between 0% and 100%, is reduced when the Vehicle is In Use, and increases when the Vehicle is Plugged to a Charging Area.
	
	\item \emph{Banking System}
	An interface that allows the System to charge the users for a Fee.
	
	\item \emph{Car}
	An electric car owned by the Car-sharing service, rented to the User and tracked by the System. A Car has a status and a Plugged flag, the status can be:
	\begin{itemize}
		\item \textit{In Use}, if a User is driving it
		\item \textit{Available}, if it can be Reserved
		\item \textit{Reserved}, if it is linked to a User but not yet In Use
		\item \textit{Unavailable} if it can't be \textit{Reserved} (for example due to damage, battery exhaustion etc.)
	\end{itemize}
	The \textit{Plugged} flag indicate if the \emph{Car} is plugged or not to the power grid.
	
	\item \emph{Car-sharing}
	A Car-sharing service allows Users to rent Cars for a limited amount of time, being charged a Fee according to time and possibly applying a Discount or an Increase.

	\item \emph{Charging Area}
	A special Parking Area where Cars can be Plugged.

	\item \emph{Company}
	The enterprise that want to built the System to provide a car-sharing services to its Users. It is the main stakeholder.

	\item \emph{Database}
	A structure that holds informations linked logically according to relationships. For instance, a Database could hold records of every registered User, every available Cars and every time a User rented a Vehicle.

	\item \emph{Discount}
	A reduction in the Fee to be paid because of good behaviour on the part of the User, e.g. leaving the Cars plugged or bringing it back with a mostly-full battery. The actions that constitute good behaviour are determined ad detailed further in the document.

	\item \emph{Employee}
	He's an employee of the Company which is charged of every kind of maintenance of the Car. (Charging the Car battery on-site, moving a Car to a Charging Area and so on). 

	\item \emph{Fee}
	The amount of money that the User will be charged for their usage of the Car-sharing service, or for making a Reservation that is not used.

	\item \emph{GPS}
	Global Positioning System. A system that allows to precisely know the position of the device on which is installed. (Detect Car's, User's and Charging Area's position).


	\item \emph{Surcharge}
	An increase in the Fee to be paid because of improper behaviour on the part of the User, e.g. bringing the Cars back with a mostly-empty battery.

	\item \emph{Management System}
	An external system that allows administrative access to the internal Database.

	\item \emph{Parking Area}
	A place where the User can leave their Car and exit it to end the Ride. Parking Areas are predefined by the System.

	\item \emph{Passenger}
	Someone (either User or not) who travels in a Cars rented by a User together with them.

	\item \emph{Plug}
	A part of the Car that interfaces itself with a Socket of a Charging Area. If a Car's Plug is linked to a Charging Area's Socket, the Car is Plugged.

	\item \emph{Ride}
	Represents the travel done with the Car by the User. It starts from the moment the User ignites the engine of the Car and ends when the Car is parked in a Parking Area and the User exits the Car.

	\item \emph{Reservation}
	A User can make a Reservation in order to link themselves with an Available Car.

	\item \emph{Socket}
	A part of the Charging Area that interfaces itself with the Plug of a Car. If a Car's Plug is linked to a Charging Area's Socket, the Car is Plugged.

	\item \emph{System}
	The automated software structure this document is about.
	It tracks Users and Vehicles and deals with all the details needed for Carsharing, from GPS mapping to charging Users with Fees.

	\item \emph{User}
	A person registered on the System, who will use the Vehicles for a Fee.

	\item \emph{Visitor}
	A person who needs to complete the registration in order to become a User.
\end{itemize}

\subsubsection{Document specific terms}
\begin{itemize}
	\item \emph{Alloy.}
	A descriptive language that allows to describe a set of structures through constraints.
	\item \emph{DBMS.}
	Data Base Management System. A software interface allowing to interact with the \emph{Database}.
	\item \emph{RASD.}
	Requirements Analysis and Specification Document. This document, describing the \emph{System} to be developed.
	\item \emph{UC.}
	Use Case. A description of interaction between \emph{User}s and \emph{System}.
	\item \emph{UML.}
	Unified Modeling Language. A language for modeling Object-Oriented software systems.
\end{itemize}