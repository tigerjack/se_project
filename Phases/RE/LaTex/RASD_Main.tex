\documentclass[12pt]{article}

\usepackage[utf8]{inputenc}

\title{Requirement Analysis and Specification Document for PowerEnJoy}
\author{Enrico Migliorini, Alessandro Paglialonga, Simone Perriello}

\setcounter{secnumdepth}{4}
%S: TOC Depth set to 3 as specified by Prof during lessons
\setcounter{tocdepth}{3}

\begin{document}
\maketitle
\clearpage
\tableofcontents

\clearpage\section{Introduction}
\subsection{Purpose}
This document presents the requirements of the \emph{PowerEnJoy} system, aimed at powering a car-sharing service. Said requirements will be presented using both natural language and diagrams.
\subsection{Intended Audience}
This document is addressed to all the stakeholders in the \emph{PowerEnJoy} project. This includes, but is not limited to, the CEO, the end users, the development committee, product designers and engineers, quality assurance and marketing.
\subsection{Product Scope}
The \emph{PowerEnJoy} is a partially automated electric car-sharing service. The system keeps track of users and Cars, addresses users to available cars, locks cars when not in use, and charges the users for use and abuse of the Cars. The system also needs to keep track of the battery level of the Cars and dispatch personnel to connect low-on-battery Cars to the power grid.

It is important to notice that this document only describes the requirements for the software dealing with the cars, not the hardware on which the software will run or the management system.
\subsection{Definitions, Acronyms and Abbreviations}
\subsubsection{Business terms glossary}
[OBSOLETE: REFER TO GLOSSARY.TXT, THIS WILL BE UPDATED LATER]
\paragraph{Car-sharing}
A \emph{Car-sharing} service allows \emph{Users} to rent \emph{Cars} for a limited amount of time, being charged a \emph{Fee} according to time and possibly applying a \emph{Discount} or an \emph{Increase}.
\paragraph{Database}
A structure that holds informations linked logically according to relationships. For instance, a \emph{Database} could hold records of every registered \emph{User}, every available \emph{Car} and every time a \emph{User} rented a \emph{Car}.
\paragraph{Discount}
A reduction in the \emph{Fee} to be paid because of good behaviour on the part of the \emph{User}, e.g. leaving the \emph{Car} plugged or bringing it back with a mostly-full battery. The actions that constitute good behaviour are determined ad detailed further in the document. [ADD A REFERENCE WHEN YOU MAKE THE SECTION]
\paragraph{Fee}
The amount of money that the \emph{User} will be charged for his usage of the \emph{Car-sharing} service.
\paragraph{Increase}
An increase in the \emph{Fee} to be paid because of improper behavious on the part of the \emph{User}, e.g. bringing the \emph{Car} back with a mostly-empty battery.
\paragraph{Management System}
An external system that allows administrative access to the internal \emph{Database}.
\paragraph{System}
The automated software structure this document is about. It tracks \emph{Users} and \emph{Cars} and deals with all the details needed for \emph{Car-sharing}, from GPS mapping to charging \emph{Users} with \emph{Fees}.
\paragraph{User}
A person registered on the \emph{System}, who will use the \emph{Cars} for a \emph{Fee}.
\paragraph{Car}
An electric car owned by the \emph{Car-sharing} service, rented to the \emph{User} and tracked by the \emph{System}.
\subsubsection{Document specific terms}
\paragraph{Alloy}
A descriptive language that allows to describe a set of structures through constraints.
\paragraph{DBMS}
Data Base Management System. A software interface allowing to interact with the \emph{Database}.
\paragraph{RASD}
Requirements Analysis and Specification Document. This document, describing the \emph{System} to be developed.
\paragraph{UC}
Use Case. A description of interaction between \emph{Users} and \emph{System}.
\paragraph{UML}
Unified Modeling Language. A language for modeling Object-Oriented software systems.

\clearpage\section{General Description}
\subsection{General Description}
The PowerEnjoy system is created in order to provide a unified platform to be used for reserving and renting electrical Cars.

Access to the full System is reserved to the registered Users, although it is also possible for a Visitor to register themselves, becoming a User.

Once registered, a User will have easy access to the functionalities of PowerEnJoy, allowing them to see the locations of nearby Cars, reserve and rent them and enable money-saving options to become eligible for Discounts.
\subsection{Product Perspective}
The main System will be housed on a central elaborator, called server. This server will be responsible for all necessary operations, including communicating with the internal Database, keeping track of all the Cars and interacting with the Banking System in order to charge Fees to the Users.
The user will access the System through a mobile app (client), connected to the main System via Internet, and interacting with the Smartphone's GPS System. Information exchange between the client and the server will be based on the HTTP protocol.
\subsection{Product Functions}
Here are explained the main needs of the PowerEnJow system.
\begin{center}
  \begin{tabular}{|p{.2\textwidth}|p{.12\textwidth}|p{.58\textwidth}|}
    \hline
    \textbf{Need} & \textbf{Priority} & \textbf{Function} \\ \hline
    Registration & ? & Allows a Visitor to register their credentials, becoming a User \\ \hline
    Login & ? & Lets a User access the system \\ \hline
    Find Nearby Cars & ? & Shows Available Cars in a location \\ \hline
    Reservation & ? & Allows a user to set a nearby Car as Reserved \\ \hline
    Proximity Unlock & ? & Unlocks the car when the User who Reserved it is close \\ \hline
  \end{tabular}
\end{center}
\begin{center}
  \begin{tabular}{|p{.2\textwidth}|p{.12\textwidth}|p{.58\textwidth}|}
    \hline
    \textbf{Need} & \textbf{Priority} & \textbf{Function} \\ \hline
    Computation & ? & Computes the Fee to be paid, including Discounts and Increases \\ \hline
    Payment & ? & Charges the user for the computed Fee \\ \hline
    Detect Parking & ? & Detects whether a Car has been left in a Parking Area \\ \hline
    Detect Charging & ? & Detects whether a Car is Plugged to a Charging Area \\ \hline
    Administrative Access & ? & Allows an administrator to modify the information in the Database or parametres of the System \\ \hline
  \end{tabular}
\end{center}
\subsection{User Classes and Characteristics}
\begin{center}
  \begin{tabular}{|p{.2\textwidth}|p{.39\textwidth}|p{.39\textwidth}|}
    \hline
    \textbf{Name} & \textbf{Description} & \textbf{Responsibility} \\ \hline
    Visitor & A person who needs to undergo Registration to become a User & Performs a Registration \\ \hline
    User & Someone who is registered in the system and can access its functionalities & Can find, reserve and rent Cars \\ \hline
    Administrator & A specialized worker for the Company & Can modify the Database and the application's various parameters \\ \hline
  \end{tabular}
\end{center}
\subsection{Operating Environment}\label{OE}
Users and Visitors will access the System through a smartphone application. The smartphones used to access the application will need to have a sufficiently advanced hardware in order to run it, and the application will require both an Internet connection and a GPS signal. The mobile application will, however, only be required in order to run the Find Nearby Cars, Reservation and Proximity Unlock functions.

The Cars will also need to be modified by adding a device capable of getting GPS signal and remotely communicate with the central System.

The central System will need a suitably powerful elaborator to run on.
\subsection{Design and Implementation Constraints}
The system will employ the HTTP protocol in order to have the client software, both the software application and the one installed on the Cars, communicate with the central System. As mentioned above, a network connection will be necessary. In order to maximize availability, it is advised to use the best available hardware.

User ID and password will be required in order to access the System. In order to register as a User, a Visitor needs to undergo Registration. 
\subsection{Assumptions and Dependencies}
[INTEGRATE FROM THE LIST]

\clearpage\section{External Interface Requirements}
The interfaces the system should interact with.
\subsection{User Interface}
The User Interface is provided by the client smartphone application, and allows the user to perform all of their actions.
\subsection{Hardware Interface}
The client application must be developed so as to have access to both the phone's network connection and its GPS locator.

The central server must be provided with one or more sufficiently advanced computers that may run the server-side application, and allow it access to a high-speed network connection.

The Cars will be fitted with the device mentioned in \ref{OE}, allowing access to various informations, such as battery status, location, and eventual presence of Passengers.
\subsection{Software Interface}
The System does not need to interface with external software. Database management can be performed via the Administrative Access function.
\subsection{Communication Interface}
As mentioned above, the System heavily uses Internet communications protocols, mainly the HTTP protocol. HTTP requests to and from the server will be mostly carried by mobile network connections.

\clearpage\section{Functional Requirements}
The functionality for the various users.
\subsection{Use cases specification}
% Counter to use for UseCaseId
\newcounter{UseCaseIdCounter}

% It starts from 0 by default, so I've incremented it.
\stepcounter{UseCaseIdCounter}

\begin{longtable}{|p{0.2\linewidth} p{0.8\linewidth}|}
	% Some table settings
	\captionsetup{labelformat=empty} % To not show Table 1, Table 2 under the table
	\caption{\textbf{Register Account}} % Table caption
	\label{UC_Register}%If later on you want to refer to this label, you can this label. 
	\\ \hline % end of row + new horizontal line
	
	% The table itself
	\textbf{ID} & UC\theUseCaseIdCounter \\ \hline
	\textbf{Description} & The \emph{Visitor} wants to create an \emph{Account} for the \emph{Car-Sharing} Service. \\ \hline
	\textbf{Actors} & \emph{Visitor}.\\ \hline
	\textbf{Pre-Conditions} & The \emph{Visitor} connects to the \emph{Company's Car-Sharing}WebSite/Application. \\ \hline
	\textbf{Flow of events} & 
		\begin{enumerate}
			\item The \emph{Visitor} selects the function \textit{\textquotedblleft{Sign Up}\textquotedblright}.
			\item The \emph{System} returns a form to enter all the required data: Name, Surname, Birth date, ID Card Number, Driving License number and Credit Card number. It also asks for an email address and a password which will be used for the future logins.
			\item The \emph{Visitor} fills the form with all the required information.
			\item The \emph{System} stores the request together with all the data provided with it, generates a random activation URL and asks the \emph{Mail System} to forward his/her URL to the email address of the \emph{Visitor}.
		\end{enumerate}	 \\ \hline
	\textbf{Post Conditions} & The \emph{Mail System} sends the activation URL to the \emph{Visitor}'s email provided in the registration form. \\ \hline
	\textbf{Exceptions} & 
		\begin{itemize}
		\item The \emph{System} recognizes invalid or missing data in the form compiled by the emph{Visitor}and informs him/her of the error. The flow of events restarts from point 1.
		\item The Visitor inserts in the form a Social Security Number, or ID Card Number, or Driving License number, or Email Address, which is already present in the System. The System shows an error message saying that some of those credentials were already been inserted into the System for another account. The flow of events restarts from point 1.
		\end{itemize} \\ \hline
\end{longtable}


\stepcounter{UseCaseIdCounter}
\begin{longtable}{|p{0.2\linewidth} p{0.8\linewidth}|}
	% Some table settings
	\captionsetup{labelformat=empty} % To not show Table 1, Table 2 under the table
	\caption{\textbf{Activate Account}} % Table caption
	\label{UC_Activate}	\\ \hline
	
	% The table itself
	\textbf{ID} & UC\theUseCaseIdCounter \\ \hline
	\textbf{Description} & The \emph{Visitor} wants to activate his/her \emph{Account}. \\ \hline
	\textbf{Actors} & \emph{Visitor}.\\ \hline
	\textbf{Pre-Conditions} & The \emph{Visitor} has received the activation URL on his/her mail box. \\ \hline
	\textbf{Flow of ev-ents} & 
	\begin{enumerate}
		\item The \emph{Visitor} opens the received activation URL.
		\item The \emph{System} acknowledges that the Visitor has arrived in his/her activation Web Page and activates his/her account.
	\end{enumerate}	 \\ \hline
	\textbf{Post Conditions} & The \emph{Visitor} is now become an \emph{User} which can access the \emph{System} using the credentials (Email, password) he provided during the registration phase. \\ \hline
	\textbf{Exceptions} & 
	\begin{itemize}
		\item The Activation URL expires after 10 days it has been generated. The Visitor?s data are cancelled from the System and the Visitor will have to perform the Registration (UC1) again.
	\end{itemize} \\ \hline
\end{longtable}

\stepcounter{UseCaseIdCounter}

\begin{longtable}{|p{0.2\linewidth} p{0.8\linewidth}|}
	% Some table settings
	\captionsetup{labelformat=empty} % To not show Table 1, Table 2 under the table
	\caption{\textbf{Log In}} % Table caption
	\label{UC_Login}%If later on you want to refer to this label, you can use this label. 
	\\ \hline % end of row + new horizontal line
	
	% The table itself
	\textbf{ID} & UC\theUseCaseIdCounter \\ \hline
	\textbf{Description} & The \emph{Visitor} wants to log in the \emph{System}. \\ \hline
	\textbf{Actors} & \emph{Visitor}.\\ \hline
	\textbf{Pre-Conditions} & The \emph{Visitor} connects to the Company's \emph{Car-Sharing WebSite/Application} \\ \hline
	\textbf{Flow of events} & 
	\begin{enumerate}
		\item The \emph{Visitor} selects the function \textit{\textquotedblleft{Login}\textquotedblright} .
		\item The \emph{System} shows the \emph{Visitor} a login form, asking him to insert the email and password provided in the registration form.
		\item The \emph{Visitor} inserts the pair (Email,Password) used during the registration phase and selects the function \textit{\textquotedblleft{Log me in}\textquotedblright}
	\end{enumerate}	 \\ \hline
	\textbf{Post Conditions} & The \emph{System} verifies the existence of an account associated with that pair (Email,password) and logs the \emph{Visitor} in. The \emph{Visitor} has now become \emph{User}  \\ \hline
	\textbf{Exceptions} & 
	\begin{itemize}
		\item The Activation URL expires after 10 days it has been generated. The Visitor?s data are cancelled from the System and the Visitor will have to perform the Registration (UC1) again.
	\end{itemize} \\ \hline
\end{longtable}
\stepcounter{UseCaseIdCounter}

\begin{longtable}{|p{0.2\linewidth} p{0.8\linewidth}|}
	% Some table settings
	\captionsetup{labelformat=empty} % To not show Table 1, Table 2 under the table
	\caption{\textbf{Log Out}} % Table caption
	\label{UC_Logout}%If later on you want to refer to this label, you can use this label. 
	\\ \hline % end of row + new horizontal line
	
	% The table itself
	\textbf{ID} & UC\theUseCaseIdCounter \\ \hline
	\textbf{Description} & The User wants to log out from the System. \\ \hline
	\textbf{Actors} & \emph{User}.\\ \hline
	\textbf{Pre-Conditions} & The \emph{User} is logged in the \emph{System} \\ \hline
	\textbf{Flow of events} & 
	\begin{enumerate}
		\item The \emph{User} selects the function \textit{\textquotedblleft{Log out}\textquotedblright} .
		\item The \emph{System} performs the \emph{User}'s logout.
	\end{enumerate}	 \\ \hline
	\textbf{Post Conditions} & The \emph{System} shows the confirmation of the logout to the \emph{User}.
	
	The \emph{User} is now not able to use the \emph{System} functionalities dedicated to Users anymore (until he logs in again). \\ \hline
	\textbf{Exceptions} &  \\
%	\begin{itemize}
%	\end{itemize} \\ 
\hline
\end{longtable}
\stepcounter{UseCaseIdCounter}

\begin{longtable}{|p{0.2\linewidth} p{0.8\linewidth}|}
	% Some table settings
	\captionsetup{labelformat=empty} % To not show Table 1, Table 2 under the table
	\caption{\textbf{Locate Available Cars}} % Table caption
	\label{UC_LocateCars}%If later on you want to refer to this label, you can use this label. 
	\\ \hline % end of row + new horizontal line
	
	% The table itself
	\textbf{ID} & UC\theUseCaseIdCounter \\ \hline
	\textbf{Description} & The \emph{User} wants to locate the avialable \emph{Cars}. \\ \hline
	\textbf{Actors} & \emph{User}.\\ \hline
	\textbf{Pre-Conditions} & The \emph{User} is logged in the \emph{System} \\ \hline
	\textbf{Flow of events} & 
	\begin{enumerate}
		\item The \emph{User} selects the function \textit{\textquotedblleft{Locate Cars}\textquotedblright}.
		\item The \emph{System} shows a text box asking the \emph{User} to provide an address near which they would like to see the \emph{Cars} whose state is \textit{Available}.
		\item The \emph{User} inserts the desired address and selects the \textit{\textquotedblleft{Locate}\textquotedblright} function.
	\end{enumerate}	 \\ \hline
	\textbf{Post Conditions} & The \emph{System} shows the \emph{User} a map containing all the \emph{Cars} whose state is Available and which are within a 5KM range from the provided address. \\ \hline
	\textbf{Alternative Flow of Events} & The \emph{User} selects the function \textit{\textquotedblleft{Near Me}\textquotedblright} instead of inserting a specific address and sends their \emph{GPS Coordinates} to the \emph{System}. \\ \hline
	\textbf{Exceptions} & 
	\begin{itemize}
		\item The System does not find the inserted address and informs the User. The Flow of Events starts from point 1.
		\item There are no available Cars in the specified address/User?s Position. The System informs the User. The Flow of Events start from point 1.
	\end{itemize} \\ \hline
\end{longtable}
\stepcounter{UseCaseIdCounter}

\begin{longtable}{|p{0.2\linewidth} p{0.8\linewidth}|}
	% Some table settings
	\captionsetup{labelformat=empty} % To not show Table 1, Table 2 under the table
	\caption{\textbf{Reserve Available Car}} % Table caption
	\label{UC_ReserveCar}%If later on you want to refer to this label, you can use this label. 
	\\ \hline % end of row + new horizontal line
	
	% The table itself
	\textbf{ID} & UC\theUseCaseIdCounter \\ \hline
	\textbf{Description} & The \emph{User} wants to reserve a \emph{Car}. \\ \hline
	\textbf{Actors} & \emph{User}.\\ \hline
	\textbf{Pre-Conditions} & The \emph{User} is logged in the \emph{System}, the \emph{User} does not have another active reservation, the \emph{User} is not driving another \emph{Car}, and the System has found available \emph{Cars} when the \emph{User} activated the \textit{\textquotedblleft{Locate Available Cars}\textquotedblright} function. \\ \hline
	\textbf{Flow of events} & 
	\begin{enumerate}
		\item The \emph{User} chooses a specific \emph{Car} among those showed on the map.
		\item The \emph{User} selects the function \textit{\textquotedblleft{Reserve this Car}\textquotedblright}.
	\end{enumerate}	 \\ \hline
	\textbf{Post Conditions} & The \emph{System} stores the \emph{Reservation} of the \emph{Car}, changing its status to \emph{Reserved}.
The \emph{System} activates a countdown of 1 hour during which the \emph{User} will have the possibility to unlock the \emph{Reserved} \emph{Car}. \\ \hline
	\textbf{Exceptions} & \\ \hline
\end{longtable}
\stepcounter{UseCaseIdCounter}

\begin{longtable}{|p{0.2\linewidth} p{0.8\linewidth}|}
	% Some table settings
	\captionsetup{labelformat=empty} % To not show Table 1, Table 2 under the table
	\caption{\textbf{Unlock Car}} % Table caption
	\label{UC_UnlockCar}%If later on you want to refer to this label, you can use this label. 
	\\ \hline % end of row + new horizontal line
	
	% The table itself
	\textbf{ID} & UC\theUseCaseIdCounter \\ \hline
	\textbf{Description} & The \emph{User} wants the \emph{System} to open the doors of the \emph{Car} in order to enter it. \\ \hline
	\textbf{Actors} & \emph{User}.\\ \hline
	\textbf{Pre-Conditions} & The \emph{User} is logged in the \emph{System} and has reserved a \emph{Car}. \\ \hline
	\textbf{Flow of events} & 
	\begin{enumerate}
		\item The \emph{User} activates the function\textit{\textquotedblleft{Unlock Car}\textquotedblright}.
		\item The \emph{User} sends their GPS coordinates to the \emph{System}.
		\item The \emph{System} checks that the GPS coordinates of the \emph{User} are within a 15 metres range from those of the \emph{Car} itself.
	\end{enumerate}	 \\ \hline
	\textbf{Post Conditions} & The \emph{System} has verified that the \emph{User} is nearby the car (within the specified distance range) and unlocks the \emph{Car}'s doors.
	 
The \emph{System} then changes the \emph{Car} status to \emph{In Use} and sets the \emph{Plugged} Flag to False.

The \emph{User} enters the \emph{Car}. \\ \hline
	\textbf{Exceptions} & If one hour has passed since the reservation has been made and the \emph{User} hasn't unlocked the Car, either because he wasn't within the 15 meters distance range or didn't activate this function, then:
	\begin{itemize}
	\item The reservation expires, so that the \emph{User} cannot unlock the \emph{Car} anymore (unless they reserve it again).
	\item The \emph{System} changes the \emph{Car}'s status to \emph{Available}.
	\item The \emph{System} communicates to the \emph{Banking System} the \emph{Fee} to charge the \emph{User} (this sum amounts to 1 EUR).
	\item The \emph{System} allows the \emph{User} to perform another reservation.
	\end{itemize} \\ \hline
\end{longtable}
\stepcounter{UseCaseIdCounter}

\begin{longtable}{|p{0.2\linewidth} p{0.8\linewidth}|}
	% Some table settings
	\captionsetup{labelformat=empty} % To not show Table 1, Table 2 under the table
	\caption{\textbf{Drive Car}} % Table caption
	\label{UC_DriveCar}%If later on you want to refer to this label, you can use this label. 
	\\ \hline % end of row + new horizontal line
	
	% The table itself
	\textbf{ID} & UC\theUseCaseIdCounter \\ \hline
	\textbf{Description} & The \emph{User} starts driving the \emph{Reserved} \emph{Car}. \\ \hline
	\textbf{Actors} & \emph{User}.\\ \hline
	\textbf{Pre-Conditions} & The \emph{User} has unlocked the doors of the \emph{Car} and entered it. \\ \hline
	\textbf{Flow of events} & 
	\begin{enumerate}
		\item The \emph{User} starts the engine of the \emph{Car}.
		\item The \emph{System} starts the Ride Timer which indicates the time usage of the \emph{Car}.
		\item {[}Extension Point UC 9{]}
		\item {[}Extension Point UC 12{]}
		\item The \emph{System} calculates the current \emph{Fee} charged to the \emph{User} (calculated as a given amount of money per minute on the Ride Timer) while showing it on the on-board screen.
	\end{enumerate}	 \\ \hline
	\textbf{Post Conditions} & The \emph{User} drives the \emph{Car} \\ \hline
	\textbf{Exceptions} & \\ \hline
\end{longtable}
\stepcounter{UseCaseIdCounter}

\begin{longtable}{|p{0.2\linewidth} p{0.8\linewidth}|}
	% Some table settings
	\captionsetup{labelformat=empty} % To not show Table 1, Table 2 under the table
	\caption{\textbf{Drive With Passengers $<<$extends UC 8$>>$}} % Table caption
	\label{UC_DriveWithPassengers}%If later on you want to refer to this label, you can use this label. 
	\\ \hline % end of row + new horizontal line
	
	% The table itself
	\textbf{ID} & UC\theUseCaseIdCounter \\ \hline
	\textbf{Description} & The \emph{User} picks up \emph{Passengers} to share the ride with. \\ \hline
	\textbf{Actors} & \emph{User}.\\ \hline
	\textbf{Pre-Conditions} & The \emph{User} is driving their \emph{Reserved} \emph{Car}. \\ \hline
	\textbf{Flow of events} & 
	\begin{enumerate}
		\item The \emph{User} picks up the \emph{Passengers}.
		\item The \emph{Car} detects the presence and number of the \emph{Passengers}.
	\end{enumerate}	 \\ \hline
	\textbf{Post Conditions} & The \emph{User} is sharing the ride with their \emph{Passengers}.
	
	The \emph{System} stores the number of \emph{Passengers} who were picked up and whether they stayed in the \emph{Car} for at least 3 minutes.\\ \hline
	\textbf{Exceptions} & \\ \hline
\end{longtable}
\stepcounter{UseCaseIdCounter}

\begin{longtable}{|p{0.2\linewidth} p{0.8\linewidth}|}
	% Some table settings
	\captionsetup{labelformat=empty} % To not show Table 1, Table 2 under the table
	\caption{\textbf{End Ride}} % Table caption
	\label{UC_EndRide}%If later on you want to refer to this label, you can use this label. 
	\\ \hline % end of row + new horizontal line
	
	% The table itself
	\textbf{ID} & UC\theUseCaseIdCounter \\ \hline
	\textbf{Description} & The \emph{User} ends the ride and the \emph{System} processes the \emph{Fee}. \\ \hline
	\textbf{Actors} & \emph{User}.\\ \hline
	\textbf{Pre-Conditions} & The \emph{User} parks the \emph{Car} in one of the \emph{Parking Areas}. \\ \hline
	\textbf{Flow of events} & 
	\begin{enumerate}
		\item The \emph{User} exits the \emph{Car}.
		\item The \emph{System} verifies that no one is in the \emph{Car}.
		\item The \emph{System} checks the \emph{Battery} status. 
		\item The \emph{System} checks, via the GPS coordinates, whether the \emph{User} has left the \emph{Car} within a 3KM distance range from the nearest \emph{Charging Area}.
		\item The \emph{System} checks if the \emph{User} drove with \emph{Passengers} (UC9).
		\item {[}Extension Point UC11{]}.
	\end{enumerate}	 \\ \hline
	\textbf{Post Conditions} & The \emph{System} locks the doors of the \emph{Car} and sets its status to \emph{Available}.
	
	The \emph{System} communicates to the \emph{Banking System} the final \emph{Fee} to charge the \emph{User}. \\ \hline
	\textbf{Alternative Flow of Events} & 
	\begin{itemize}
		\item The \emph{Battery} status is higher than 50\%, the \emph{User} didn't or did take at least 2 \emph{Passengers} with him for at least 3 minutes (UC9), didn't leave the \emph{Car} further than 3KM from the nearest \emph{Charging Area}, didn't plug the \emph{Car} (UC11), hence the System applies a 20\% \emph{Discount} on the \emph{Fee} of the last ride and communicates it to the \emph{Banking System} the \emph{Fee} which will be charged to the \emph{User}.
		\item The \emph{User} did plug the \emph{Car} (UC11), the \emph{Battery} status is higher than or equal to 20\%, they didn't or did take at least 2 \emph{Passengers} with him for at least 3 minutes (UC9), hence the System applies a 30\% \emph{Discount} on the \emph{Fee} of the last ride and communicates to the \emph{Banking System} the \emph{Fee} which will be charged to the \emph{User}.
		\item The \emph{User} did plug the \emph{Car} (UC11), the \emph{Battery} status is lower than 20\%, they didn't or did take at least 2 \emph{Passengers} with him for at least 3 minutes (UC9), hence the System doesn't apply any \emph{Discount} or \emph{Surcharge} on the \emph{Fee} of the last ride and communicates to the \emph{Banking System} the \emph{Fee} which will be charged to the \emph{User}.
		\item The \emph{User} didn't plug the \emph{Car} (UC11), the \emph{Battery} status is higher than 50\%, they either did or didn't take at least 2 \emph{Passengers} with him for at least 3 minutes (UC9), did leave the \emph{Car} further than 3KM from the nearest \emph{Charging Area}, hence the System applies a 10\% \emph{Surcharge} on the \emph{Fee} of the last ride and communicates to the \emph{Banking System} the \emph{Fee} which will be charged to the \emph{User}.
		% Even longtable can't handle tables this long
		\end{itemize} \\ &
		\begin{itemize}
		\item The \emph{Battery} status is between 20\% and 50\% (included), the \emph{User} did take at least 2 \emph{Passengers} with him for at least 3 minutes (UC9), didn't leave the \emph{Car} further than 3KM from the nearest \emph{Charging Area}, didn't plug the \emph{Car} (UC11), hence the System applies a 10\% \emph{Discount} on the \emph{Fee} of the last ride and communicates to the \emph{Banking System} the \emph{Fee} which will be charged to the \emph{User}.
		\item The \emph{Battery} status is lower than 20\%, the \emph{User} did take at least 2 \emph{Passengers} with him for at least 3 minutes (UC9), either did or didn't leave the \emph{Car} further than 3KM from the nearest \emph{Charging Area}, didn't plug the \emph{Car} (UC11), hence the System applies a 20\% \emph{Surcharge} on the \emph{Fee} of the last ride and communicates to the \emph{Banking System} the \emph{Fee} which will be charged to the \emph{User}.
		\item The \emph{Battery} status is between 20\% and 50\% (included), the \emph{User} did take at least 2 \emph{Passengers} with him for at least 3 minutes (UC9), did leave the \emph{Car} further than 3KM from the nearest \emph{Charging Area}, hence the System applies a 20\% \emph{Surcharge} on the \emph{Fee} of the last ride and communicates to the \emph{Banking System} the \emph{Fee} which will be charged to the \emph{User}.
		\item The \emph{Battery} status is lower than 20\%, the \emph{User} didn't take at least 2 \emph{Passengers} with him for at least 3 minutes (UC9), either did or didn't leave the \emph{Car} further than 3KM from the nearest \emph{Charging Area}, didn't plug the \emph{Car} (UC11), hence the System applies a 30\% \emph{Surcharge} on the \emph{Fee} of the last ride and communicates to the \emph{Banking System} the \emph{Fee} which will be charged to the \emph{User}.
		\end{itemize} \\ &
		\begin{itemize}
		\item The \emph{Battery} status is higher than 50\%, the \emph{User} either did or didn't take at least 2 \emph{Passengers} with him for at least 3 minutes (UC9), did leave the \emph{Car} further than 3KM from the nearest \emph{Charging Area}, didn't plug the \emph{Car} (UC11), hence the System applies a 10\% \emph{Surcharge} on the \emph{Fee} of the last ride and communicates to the \emph{Banking System} the \emph{Fee} which will be charged to the \emph{User}.	
		\item The \emph{Battery} status is between 20\% and 50\% (included), the \emph{User} didn't take at least 2 \emph{Passengers} with him for at least 3 minutes (UC9), did leave the \emph{Car} further than 3KM from the nearest \emph{Charging Area}, hence the System applies a 30\% \emph{Surcharge} on the \emph{Fee} of the last ride and communicates to the \emph{Banking System} the \emph{Fee} which will be charged to the \emph{User}.
	\end{itemize} \\ \hline
	\textbf{Exceptions} & If the \emph{Battery} status reaches 0\% of capacity or the \emph{Car} detects a major damage, the \emph{Car} stops and an assistance team is deployed. \\ \hline
\end{longtable}
\stepcounter{UseCaseIdCounter}

\begin{longtable}{|p{0.2\linewidth} p{0.8\linewidth}|}
	% Some table settings
	\captionsetup{labelformat=empty} % To not show Table 1, Table 2 under the table
	\caption{\textbf{Plug the Car $<<$extends UC 10$>>$}} % Table caption
	\label{UC_PlugCar}%If later on you want to refer to this label, you can use this label. 
	\\ \hline % end of row + new horizontal line
	
	% The table itself
	\textbf{ID} & UC\theUseCaseIdCounter \\ \hline
	\textbf{Description} & The \emph{User} plugs the \emph{Car} for recharging. \\ \hline
	\textbf{Actors} & \emph{User}.\\ \hline
	\textbf{Pre-Conditions} & The \emph{User} has parked the \emph{Car} in one of the \emph{Charging Areas} designated by the \emph{System}. \\ \hline
	\textbf{Flow of events} & 
	\begin{enumerate}
		\item The \emph{User} plugs the \emph{Car} into a \emph{Socket} of the \emph{Charging Area}. 
		\item The \emph{System} detects that the \emph{Car} has been plugged within 2 minutes since the \emph{User} got off the \emph{Car}.
	\end{enumerate}	 \\ \hline
	\textbf{Post Conditions} & The \emph{Battery} of the \emph{Car} is charging and the \emph{System} remembers the \emph{User}'s action for possible discounts.
	
The \emph{System} sets the \emph{Car}'s \emph{Plugged} flag to True.
	\\ \hline
	\textbf{Exceptions} & \\ \hline
\end{longtable}
\stepcounter{UseCaseIdCounter}

\begin{longtable}{|p{0.2\linewidth} p{0.8\linewidth}|}
	% Some table settings
	\captionsetup{labelformat=empty} % To not show Table 1, Table 2 under the table
	\caption{\textbf{Enable Money Saving Option $<<$extends UC 8$>>$}} % Table caption
	\label{UC_PlugCar}%If later on you want to refer to this label, you can use this label. 
	\\ \hline % end of row + new horizontal line
	
	% The table itself
	\textbf{ID} & UC\theUseCaseIdCounter \\ \hline
	\textbf{Description} & The \emph{User} asks the \emph{System} to suggest them a \emph{Charging Area} where to leave the \emph{Car}. \\ \hline
	\textbf{Actors} & \emph{User}.\\ \hline
	\textbf{Pre-Conditions} & The \emph{User} enables the \textit{\textquotedblleft{Money Saving}\textquotedblright} option. \\ \hline
	\textbf{Flow of events} & 
	\begin{enumerate}
		\item The \emph{System} asks the\emph{User} the destination address, providing them with a text box where to insert it.
		\item The \emph{User} provides the address to the \emph{System}.
	\end{enumerate}	 \\ \hline
	\textbf{Post Conditions} & The \emph{System} uses an algorithm which takes in consideration the distribution of the cars in the city, the final destination of the \emph{User} and the availability of power plugs in \emph{Charging Areas} to find the optimal \emph{Charging Area} for the \emph{User} to park their \emph{Car} in. The result of this algorithm will be sent to the \emph{User}, providing him the address of the \emph{Charging Area} where to leave the \emph{Car}. The \emph{User} will still have to plug the \emph{Car} in order to get a discount.
	\\ \hline
	\textbf{Exceptions} & If the \emph{Socket} of the \emph{Charging Area} has no more available plugs , the \emph{System} informs the \emph{User} and the Flow of Events starts from point 1.\\ \hline
\end{longtable}
\stepcounter{UseCaseIdCounter}
\subsection{Use Case Diagram}
\subsection{Class Diagram}
\subsection{Activity Diagrams}
\subsection{Sequence Diagrams}
\subsection{Alloy representation of requirements}

\clearpage\section{Non-functional Requirements}
Additional requirements that may be added to improve on the program.
\subsection{Performance Requirements}
\subsection{Safety and Security Requirements}
\subsection{Software Quality Attributes}
\subsubsection{Availability}
\subsubsection{Reliability}
\subsubsection{Extensibility}
\subsubsection{Interoperability}
\subsection{Business Rules}

\end{document}